%--------------------
% Packages
% -------------------
\documentclass[11pt,a4paper]{article}
\usepackage[utf8x]{inputenc}
\usepackage[T1]{fontenc}
%\usepackage{gentium}
\usepackage{mathptmx} % Use Times Font


\usepackage[pdftex]{graphicx} % Required for including pictures
\usepackage[swedish]{babel} % Swedish translations
\usepackage[pdftex,linkcolor=black,pdfborder={0 0 0}]{hyperref} % Format links for pdf
\usepackage{calc} % To reset the counter in the document after title page
\usepackage{enumitem} % Includes lists
\usepackage{dirtree}
\usepackage{listings}
\lstset{basicstyle=\ttfamily}


\frenchspacing % No double spacing between sentences
\linespread{1.2} % Set linespace
\usepackage[a4paper, lmargin=0.1666\paperwidth, rmargin=0.1666\paperwidth, tmargin=0.1111\paperheight, bmargin=0.1111\paperheight]{geometry} %margins
%\usepackage{parskip}

\usepackage[all]{nowidow} % Tries to remove widows
\usepackage[protrusion=true,expansion=true]{microtype} % Improves typography, load after fontpackage is selected

\usepackage{lipsum} % Used for inserting dummy 'Lorem ipsum' text into the template


%-----------------------
% Set pdf information and add title, fill in the fields
%-----------------------
\hypersetup{ 	
pdfsubject = {Research Computing Coursework},
pdftitle = {Research Computing Coursework},
pdfauthor = {Laura Just Fung (lj441)}
}

%-----------------------
% Begin document
%-----------------------
\begin{document} %All text i dokumentet hamnar mellan dessa taggar, allt ovanför är formatering av dokumentet

\section{Introduction}
This report discusses development of the package designed to implement automatic differentiation using dual numbers in Python. 

\section{Repository}
\subsection{Structure}
Good practice suggests the following directory structure:

\dirtree{%
.1 dual{\_}autodiff.
.2 dual{\_}autodiff.
.3 {\_}{\_}init{\_}{\_}.py.
.3 dual.py.
.2 tests.
.3 test{\_}dual.py.
.2 docs.
.3 ....
.2 build.
.3 ....
.2 source.
.3 ....
.2 pyproject.toml.
.2 setup.py.
.2 README.md.
.2 Makefile.
.2 LICENSE.
}

\subsection{Project configuration}
The following is the contents of the \texttt{pyproject.toml} file.
\lstinputlisting[language=Fortran]{pyproject.toml}

\end{document}
